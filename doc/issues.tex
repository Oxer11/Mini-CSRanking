%This file contains the LaTeX code of my laboratory report for my Database course.
%Author: 周芯怡/Xinyi Zhou <17307130354@fudan.edu.cn>
%Author: 张作柏/Zuobai Zhang <17300240035@fudan.edu.cn>

\section{技术难点}

本节主要列举一下我们在完成项目过程中遇到的问题和所涉及的技术难点。

\begin{enumerate}
\item {\bf 网站搭建}:我们使用的是Django框架,整个网站的搭建流程如下:首先,我们在urls.py中设置好每个页面的URL网址;然后,创建若干html模板,对应于这些页面;之后,书写后端函数,通过渲染模板完成响应。为了实现各个页面之间的跳转,我们在每个页面中加入超链接,然后在访问某个页面时,通过GET请求将相应的参数传给后端。这里使用GET请求的原因是,参数可以直接在URL中看到,方便调试,也方便直接访问。

\item {\bf 字符编码}:因为在url中显示的字符会进行url编码,前后端交互时如果不进行编码和解码就会出现问题。比如当url参数中带有\&时,就必须在前端的变量后加上|urlencode;前端发送给后端的字符串,有时也需要用urllib.parse.unquote()进行解码。

\item {\bf 分页处理}:像搜索结果、排行榜之类的数据条目较多的页面,我们需要使用分页技术来控制每页显示的条目数量。这里我们使用的是Django自带的paginate方法,我们只需要传入相应的参数(进行分割的查询,每个页面的条目个数,当前是第几页)就可以接受当前页面的结果。使用Django提供的接口,这种分页方式可以不需要每次都将所有的结果都返回,而只返回需要显示的结果,大大提升了查询效率。

\item {\bf 局部刷新}:在设计follow功能时,我们在每个学者的主页添加了follow按钮。当用户未登录或未关注该学者时,按钮显示文字为follow;当用户已关注该学者时,按钮显示文字为unfollow。那么,当用户点击follow时,我们要发送请求到后端,为用户添加关注,然后将follow按钮变成unfollow选项。

这一过程看似简单,但实现起来却相当麻烦。一般而言,我们点击按钮后,会发送一个GET或POST请求给后端,后端处理后会重新渲染页面,这时网页会自动刷新。然而,整个页面的显示只有follow文字这一处发生了变化,刷新整个页面效率低下,导致使用不便捷。在这里,我们使用了{\bf jquery}的局部更新工具。

我们自行书写了一小段js代码来处理这一操作,大致思路是:首先我们为follow\_button添加事件响应函数,在follow\_button被点击时触发。接着,我们会获取按钮中文字信息,并将此作为参数,向后端发送POST请求。之后,后端处理该请求,并将相应的返回值以JSON格式传给前端。最后,前端接受后端的响应,并将相应的文字信息写入按钮中。

与之相适应的,后端的代码也需要做相应的调整:当后端接收到请求后,首先判断请求的类型是GET还是POST,若是GET,这说明是常规的请求;若是POST,则说明是点击follow按钮后,jquery发送的请求。若为POST,则首先判断用户是否登录,在根据登录情况对表进行相应修改,最后将需要在按钮中显示的信息返回。

\item {\bf 排行榜显示}:排行榜页面的显示中有三个难点:{\bf 显示/隐藏教师信息}、{\bf 自动勾选领域信息}和{\bf 排行榜的局部刷新}。
\begin{itemize}
\item 访问ranklist页面时,后端会将学校的排名连同该校老师的排名一同返回给前端,而为了不然前端显示的过于冗杂,我们默认会隐藏老师的信息。当用户点击学校旁的三角按钮时,才会显示出该校的老师。这一操作是通过html中display:none的属性完成的。在html中,将一个元素的display属性设置为none后,它将完全隐藏,不占任何位置,这样后端将老师信息传递给前端后,所有的显示任务就全部交给前端了。前端要做的就是利用js,为三角按钮设置事件,当点击该按钮时,切换老师模块的display属性,并更改三角按钮的样式。
\item 自动勾选领域信息是指,为方便用户选择领域,用户可以通过on/off按钮直接勾选/取消勾选某一方向的所有领域。这一功能利用jquery的元素选择器可以简便地完成,添加事件:当点击on时,将相应方向的每个checkbox属性都设置为勾选。
\item 排行榜的局部刷新基本原理与上一条相同。两个不同点是:1.提交POST请求时,需要获取所有checkbox的勾选状况,这个我们要通过jquery获取每个checkbox的情况写入一个数组,之后将数组作为参数传入。2.因为ranklist变化较大,所以后端返回时,我们需要按照模板将ranklist的部分重新渲染一遍,然后在前端通过.html()方法重新该部分的代码。
\end{itemize}

\item {\bf 大数据处理}:因为我们的网站中涉及的数据量较大,所以查询的效率难免会受到限制,这也是在设计网站时困扰我们的一点。一种比较合理的处理方式是,允许一些数据的冗余,比如提前计算并维护pub\_cnt这种修改次数较少,但经常被用于查询,且查询效率很低的属性。虽然会增大所占用的空间,但是考虑到其对查询效率的提升,这点空间还是值得的。
\end{enumerate}


