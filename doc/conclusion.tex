%This file contains the LaTeX code of my laboratory report for my Database course.
%Author: 周芯怡/Xinyi Zhou <17307130354@fudan.edu.cn>
%Author: 张作柏/Zuobai Zhang <17300240035@fudan.edu.cn>

\section{结语}

本项目到此便已圆满完成,在这个项目中确实学到了许多课堂上学不到的数据库知识,也提升了自己的项目开发能力。有几个简单的感想:
\begin{itemize}
\item 现实生活中的关系模式远比课本中的更加复杂。在创建关系模式的时候,为了达到更高的范式要求,我们不得不将一些本来非常自然地关系拆分出来,就比如Conference\_Area等等。这类拆分虽然能减少局部依赖、传递依赖的产生,但也为我们的查询带来了不便,甚至可能会降低查询的效率。
\item 查询效率与数据冗余之间往往存在权衡(tradeoff)。虽然关系的合理拆分能提高关系模式的范式,但是过多的表导致我们不得不在查询中大量使用连接操作,这是效率非常低下的。相比之下,如果将冗余的程度放宽,或许能减少一些连接操作,进而提高我们的查询效率。
\item 我们所做的项目主要以搜索功能为主,所以对增删改操作支持的并不是很多,这是因为“搜索引擎”类的网站天然地不支持增删改操作,这些操作涉及到的需要维护的量太多,且会造成许多不协调的问题出现,所以最好的方法还是从后端批量导入数据。尽管如此,这类“搜索引擎”项目在查询的复杂度上却是大大地超出了一般地管理系统,这一点也可以从我们第5节如此长的篇幅中体现。
\end{itemize}

最后,感谢队友愿意再次和我组队!面对工作量如此巨大的Project,也正是队友的鼓励与支持,才给了我坚持下来的动力!

以及,十分感谢助教愿意检查我们的项目,提出宝贵的建议,并不厌其烦地读完这份冗长的报告!