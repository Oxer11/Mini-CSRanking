%This file contains the LaTeX code of my laboratory report for my Database course.
%Author: 周芯怡/Xinyi Zhou <17307130354@fudan.edu.cn>
%Author: 张作柏/Zuobai Zhang <17300240035@fudan.edu.cn>

\section{概览}

本项目的所有源代码公开于https://github.com/Oxer11/Mini-CSRanking。

\subsection{项目要求}

本项目需要我们设计一个数据库应用,采用客户端/服务器的结构,具体要求如下:
\begin{itemize}
\item 有完整的前后端架构(前端界面+后台数据库)
\item 有用户注册登录系统
\item 内容有实际意义且完整
\item 支持前端对数据的增删查改操作
\end{itemize}

\subsection{设计动机}

这个项目的idea是受CS Rankings\footnote{http://csrankings.org}这个网站的启发。这个网站根据发表论文的数量对所有的学校进行了排名,这为学校提供了较为客观的评价指标。同时,在学校下方,还可以看到近年发表论文数量较多的学者,可以直接进入学者的主页,这也为不少出国党的同学提供了便利,可以更方便的选择自己中意的导师。

然而,这个网站还有一些小小的问题,就是对检索功能的支持还不够流畅。它无法直接按老师或学校的名字来检索,而是需要在排行榜上进行寻找,这是很费时的一件事。相比之下,GoogleScholar的检索功能就更加友好,可以直接检索学者或论文。并且, GoogleScholar可以浏览到学者近年来发表的论文的详细信息,可以说是一个较为全面的学术网站。

同时,若我们想了解某个领域的知名学者或顶级会议,仍缺乏比较合理的手段。虽然DBLP中提供了按会议检索的功能,但是没有根据定量的指标指出某领域的知名学者。

我们的应用旨在开发一个更全面的学术网站:结合三者的长处,选取其中核心功能,并增加一些个性化的设计与美观的可视化界面。

\subsection{技术栈}

\begin{table}[h]
\begin{tabular}{ll}
{\bf 开发语言}   & JavaScript / HTML / CSS / Python / SQLite \\
{\bf 浏览器环境} & Chrome / Firefox / Safari \\
{\bf 第三方库}   & jQuery / Bootstrap                             
\end{tabular}
\end{table}

\subsection{工作流程}

我们两人在上学期的ICS课程中已经合作过一次,所以也就延续了上学期的“不分工”策略。所谓“不分工”,是指不明确区分前后端开发,而是两人共同开发,遇到具体任务时再分清两人的工作。之所以不区分前后端,一是因为项目较小,区分前后端会影响两人的交互性,导致开发效率降低;二是因为前后端的工作量差异较大,不便分工。实践证明,“不分工”策略是成功的,两人不仅顺利地完成了PJ,且两人分得的工作量与工作难度也都较为平均。我们的具体分工如下:

\noindent
{\bf 周芯怡}:
\begin{itemize}
\item 搜索功能前端页面的搭建与链接
\item 用户登录/注册功能
\item 用户个人中心的前后端设计
\item 评论消息提醒的功能设计
\item 网站的美化与运行测试
\end{itemize}
{\bf 张作柏}:
\begin{itemize}
\item 数据的搜集与处理
\item 搜索功能关系模式的设计与后端搭建
\item 使用外部元素,美化前端
\item Ranklist的前后端设计
\item 新论文的添加与消息提醒功能
\end{itemize}

\subsection{阅读指南}

本报告长达近30页,要读完实属不易,在此整理概括各部分内容,以帮助读者阅读。在第二节中,我们将简单介绍Mini-CS Ranking项目的功能特点,并展示我们的前端界面。在第三节中,我们将列举分析数据库的关系模式。数据量大是本项目的特点之一,所以第四节我们将介绍数据的采集与导入流程。在第五节中,我们将详尽地分析项目中所涉及的增删查改操作。第六节中,我们将展开讨论在项目实现过程中遇到的困难和值得留意的实现细节。最后,在第七节中,我们简单总结了完成整个项目的感想与体会。

