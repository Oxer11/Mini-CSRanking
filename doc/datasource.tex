%This file contains the LaTeX code of my laboratory report for my Database course.
%Author: 周芯怡/Xinyi Zhou <17307130354@fudan.edu.cn>
%Author: 张作柏/Zuobai Zhang <17300240035@fudan.edu.cn>

\section{数据集获取与导入}

数据量大是本项目的特征之一,单是数据库中的论文数量就达到了5万的数量,而论文作者关系更是达到了20万的级别,而获取这些数据并导入数据库,着实花费了不少精力,因此本节将重点介绍数据集的获取过程,以及根据关系模式导入数据的流程。

本节所介绍的所有代码与文件均包含在DataProcess文件夹下,独立于整个项目。

\subsection{CS Rankings开源数据}

因为CS Rankings本身是开源项目,其数据集也是公开的\footnote{https://github.com/emeryberger/CSrankings},所以我们的一部分数据直接使用了这些开源数据,以减少工作量。

这一类数据主要包括csrankings.csv和institution.csv两个文件。csrankings.csv包含了学者姓名、所在学校、学者主页和ScholarID;institution.csv包含了学校的全称。其中学者与所在学校之间的关系,如果自己获取的话会比较麻烦。因为要到相应的学校网站上去爬取老师的数据,而这类通用性较强的爬虫是比较难做的,因此我们选择直接使用CS Rankings已经整理好的开源数据。

\subsection{DBLP开源数据集解析}

众所周知,DBLP因其便利的开源数据集而闻名,许多数据挖掘相关的研究也是在DBLP的数据集上展开的,所以我们在此就使用了DBLP的开源数据集\footnote{http://dblp.uni-trier.de/xml/},来获取论文信息与作者信息。

但是因为DBLP的数据集是xml格式,所以无法直接使用的,需要先对其进行解析。解析的代码在parser.py中,作用是将XML格式的DBLP数据集,转换为便于我们处理的paper.csv和paper\_author.csv文件。

解析的过程我们参考了https://blog.csdn.net/shuaishuai3409/article/details/54316214一文。我们调用了Python标准库中的SAX解析器进行解析,其原理是在解析XML的过程中,会触发事先由用户定义的事件。我们要做的工作就是重新定义一个xml.sax.ContentHandler的类,其中包含三种事件:
\begin{enumerate}
\item 元素开始事件startElement:遇到开始标签时触发,此时可将类中的变量清零,并记录当前标签的类别。
\item 内容处理事件characters:处理一个标签中的内容,将标签中的内容保存到局部变量中。
\item 元素结束事件endElement:遇到结束标签时触发,将局部变量中的内容保存到全局的字典中。
\end{enumerate}
需要注意的是,在处理时要区分父标签和子标签,即在父标签结束时才将局部变量清零,子标签结束时则只需要更新局部变量的属性。

因为DBLP的数据集实在太大,所以我们只取了近五年中发表在部分顶会上的论文。

具体解析过程,请参见parser.py。

\subsection{爬虫获取数据}

除去以上两个数据来源,还有一部分数据是由我们自己爬取的,例如institution\_homepage.csv,其中包含了每所学校的主页网址。

爬虫的原理是,通过一个Python脚本,不断地对某些URL发送request,接受返回的html代码,并通过分析处理获取需要的内容。为了获取每所学校的首页,我们可以在Google上搜索“CS 学校名字”,一般搜索结果的第一项就是该校计算机系的首页,因此核心思路就是:利用request向Google发送搜索请求,然后从返回的第一个搜索结果中解析出学校首页的URL。
\begin{enumerate}
\item Google提供了比较简洁的搜索URL接口:\\
https://www.google.com.hk/search?q=<keyword>\&num=1\\
表示设置搜索关键词为<keyword>,返回结果数量为1。利用Python的requests库进行访问,并通过.text属性获取html内容。
\item 对html代码的解析,我们可以先利用request获取一份html代码,并保存在本地后用浏览器打开,在调试器中找到我们想要的元素的标签。之后,我们使用了Python的BeautifulSoup库,这里我们可以很便捷地通过find方法根据标签类型和标签id找到相应元素,然后通过正则表达式进行字符串的匹配和处理。
\end{enumerate}

需要注意的是,Google的反爬虫机制做的相当智能,我们必须通过减慢访问速度和切换UserAgent来避免出现人机验证界面。

具体爬虫过程,请参见scraper.py。

\subsection{数据导入}

因为调试阶段需要多次重写数据库,而数据的导入工作又十分繁琐,所以我们自己写了一个脚本专门用于导入数据。

所有已经预处理好的数据,都在DataProcess目录下的.csv文件中。我们的任务就是从.csv文件中,按表中格式解析数据,并通过get\_or\_create方法创建新数据。这里之所以使用get\_or\_create而非create,主要是因为一次性导入全部数据耗时较长,有时我们需要多次运行脚本,而这样就有重复导入数据的情况出现,通过get\_or\_create就可以避免因为插入重复元组而报错的问题。

