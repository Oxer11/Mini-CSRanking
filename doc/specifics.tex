%This file contains the LaTeX code of my laboratory report for my Database course.
%Author: 周芯怡/Xinyi Zhou <17307130354@fudan.edu.cn>
%Author: 张作柏/Zuobai Zhang <17300240035@fudan.edu.cn>

\section{设计细节}

本节主要介绍各个页面的设计细节,包括其中的涉及的难点与SQL语句。在实际的设计中,借助Django提供的便利的接口和Python对列表、字典等数据结构的支持,我们并不需要直接使用SQL语句进行查询,这样在一定程度上可以提高效率。但为了将查询功能展示的更清晰,我们还是将所涉及的SQL语句列在这里。

\subsection{信息检索}

\subsubsection{论文主页}
我们为每篇论文设计了单独的主页,其中包括的功能有显示论文作者、所属会议、所属领域:
\begin{itemize}
\item 作者:
\begin{lstlisting}[language=SQL]
select scholar_name
from Scholar_Paper
where paper_title = :pid;
\end{lstlisting}
\item 所属领域:
\begin{lstlisting}[language=SQL]
select area
from Conference_Area, Paper
where pid = :pid and Paper.conf_id = Conference_Area.conf_id;
\end{lstlisting}
\item 所属会议:因为会议是论文的一个属性,所以无需使用select语句筛选。
\begin{lstlisting}[language=SQL]
select conf_id
from Paper
where pid = :pid;
\end{lstlisting}
\end{itemize}

\subsubsection{学者主页}
\label{sec:scholar}
在每位学者的主页中,我们实现了如下功能:
\begin{itemize}
\item 显示所属学校、DBLP、GoogleScholar、Homepage等信息.
\begin{lstlisting}[language=SQL]
select nameaffiliation, DBLP, GoogleScholar, homepage
from Scholar
where name = :sid;
\end{lstlisting}
\item 列举其发表的所有论文,按时间顺序降序排序,使用如下SQL语句进行查询,查询出的表记作Papers。
\begin{lstlisting}[language=SQL]
select P.title, P.href, P.conf_id
from Scholar_Paper as SP, Paper as P, Conference as C
where SP.scholar_name = :sid and SP.paper_title = P.pid and P.conf_id = C.cid
order by C.year DESC;
\end{lstlisting}
\item 显示其所属领域
\begin{lstlisting}[language=SQL]
select distinct name
from Area as A, Conference_Area as CA, Papers as P
where CA.conf_id = P.conf_id and CA.area = A.name;
\end{lstlisting}
\item 显示与其合作过的学者
\begin{lstlisting}[language=SQL]
select distinct A.name
from Scholar_Paper as A, Scholar_Paper as B
where A.paper_title = B.paper_title and B.scholar_name = :sid;
\end{lstlisting}
\item 显示其近五年中每年发表的论文数量
\begin{lstlisting}[language=SQL]
select C.year, count(*)
from Scholar_Paper as SP, Paper as P, Conference as C
where SP.scholar_name = :sid and SP.paper_title = P.pid and P.conf_id = C.cid
group by C.year;
\end{lstlisting}
\item 关注功能
\begin{lstlisting}[language=SQL]
insert into User_Scholar(user, sch, new_paper)
values(:user, :sid, False);
\end{lstlisting}
\item 取消关注功能
\begin{lstlisting}[language=SQL]
delete from User_Scholar
where user = :user and sch = :sid;
\end{lstlisting}
\end{itemize}
注:在导入学者数据时,可以预先将学者的pub\_cnt计算导入。因为论文的增加和删除数量非常少,pub\_cnt很少变化,所以我们可以预先将其计算好存储起来。计算使用的SQL语句如下:
\begin{lstlisting}[language=SQL]
update Scholar
set pub_cnt = (
select count(*)
from Scholar_Paper as SP
where SP.scholar_name = Scholar.name
);
\end{lstlisting}

\subsubsection{学校主页}
\label{sec:institution}
在每所学校的主页中,我们实现了如下功能:
\begin{itemize}
\item 检索该校的所有学者,按发表论文数降序排序。
\begin{lstlisting}[language=SQL]
select name
from Scholar
where affiliation = :ins
order by pub_cnt DESC;
\end{lstlisting}
\item 显示学者的基本信息。查询语句同~\ref{sec:scholar}节。
\item 显示学者所属的领域。查询语句同~\ref{sec:scholar}节。
\item 显示学者最近发表的几篇论文,若查询结果数目大于5,则仅取前5项。查询语句同~\ref{sec:scholar}节。
\item 统计每个领域中发表论文的数量
\begin{lstlisting}[language=SQL]
select CA.area, count(*)
from Paper as P, Conference as C, Conference_Area as CA
where P.conf_id = C.cid and CA.conf_id = C.id
group by CA.area;
\end{lstlisting}
\item 关注功能
\begin{lstlisting}[language=SQL]
insert into User_Institution(user, ins)
values(:user, :ins);
\end{lstlisting}
\item 取消关注功能
\begin{lstlisting}[language=SQL]
delete from User_Institution
where user = :user and ins = :ins;
\end{lstlisting}
\end{itemize}
注:学校的pub\_cnt不是指发表的论文数,而是每人每篇论文统计一次。也就是说,假如一篇论文中有两位来自A校的作者,那么A校的pub\_cnt会增加2。之所以这样设置,是因为考虑到每位作者对论文都有其贡献,假如A校的5位学者和B校的1位学者合著了一篇论文,那么如果说该论文对A、B校的pub\_cnt贡献都为1,未免对A校有失公平,因为A校在论文的发表中明显贡献会更大,按论文人次统计很好的解决了这一问题。

与学者的pub\_cnt类似,学校的pub\_cnt也可以在最初计算后存储。该操作涉及的SQL语句为:
\begin{lstlisting}[language=SQL]
update Insititution
set pub_cnt = (
select sum(S.pub_cnt)
from Scholar as S
where S.affiliation = Insitution.name
);
\end{lstlisting}

\subsubsection{领域主页}
在每个领域的主页中,我们实现了如下功能:
\begin{itemize}
\item 筛选该领域中的顶会
\begin{lstlisting}[language=SQL]
select conf_id
from Conference_Area
where area = :area;
\end{lstlisting}
\item 列举每个顶会的几篇论文
\begin{lstlisting}[language=SQL]
select title, href
from Paper
where conf_id = :cid;
\end{lstlisting}
\item 筛选该领域最顶尖的学者,取pub\_cnt最高的五位
\begin{lstlisting}[language=SQL]
select name, homepage, DBLP, GoogleScholar, affiliation, pub_cnt
from Scholar as S
where exists (
select *
from Scholar_Papae as SP, Paper as P, Conference as C, Conference_Area as CA
where SP.scholar_name = S.name and SP.paper_title = P.pid and P.conf_id = C.cid and CA.conf_id = C.cid and CA.area = :area
)
order by pub_cnt DESC;
\end{lstlisting}
\item 显示这些学者的相关信息,查询语句同~\ref{sec:institution}节。
\item 统计每个顶会每年的论文发表量
\begin{lstlisting}[language=SQL]
select cid, count(*)
from Conference as C, Paper as P
where P.conf_id = C.cid
group by cid
\end{lstlisting}
\item 关注功能
\begin{lstlisting}[language=SQL]
insert into User_Area(user, area)
values(:user, :area);
\end{lstlisting}
\item 取消关注功能
\begin{lstlisting}[language=SQL]
delete from User_Area
where user = :user and area = :area;
\end{lstlisting}
\end{itemize}

\subsubsection{会议主页}
在每个领域的主页中,我们实现了如下功能:
\begin{itemize}
\item 列举会议基本信息
\begin{lstlisting}[language=SQL]
select name, abbr, year, DBLP, Href
from Conference
where cid = :cid;
\end{lstlisting}
\item 列举论文列表
\begin{lstlisting}[language=SQL]
select pid, title, href
from Paper
where conf_id = :cid;
\end{lstlisting}
\item 关注功能
\begin{lstlisting}[language=SQL]
insert into User_Conference(user, conf)
values(:user, :cid);
\end{lstlisting}
\item 取消关注功能
\begin{lstlisting}[language=SQL]
delete from User_Conference
where user = :user and conf = :cid;
\end{lstlisting}
\end{itemize}

\subsubsection{检索页面}

我们设计了一个搜索框,用于检索相关信息,涉及到的功能有:

{\bf 检索学者} 按照关键词检索学者,按pub\_cnt降序排序:
\begin{lstlisting}[language=SQL]
select name, homepage, DBLP, GoogleScholar, affiliation, pub_cnt
from Scholar
where name like '%<key>%'
order by pub_cnt DESC;
\end{lstlisting}

{\bf 检索论文}
\begin{lstlisting}[language=SQL]
select title, href, conf_id
from Paper
where title like '%<key>%';
\end{lstlisting}

{\bf 检索学校}
\begin{lstlisting}[language=SQL]
select name, homepage, pub_cnt
from Institution
where name like '%<key>%'
order by pub_cnt DESC;
\end{lstlisting}

{\bf 检索会议}
\begin{lstlisting}[language=SQL]
select name, abbr, year, DBLP, Href
from Conference
where name like '%<key>%' or abbr like '%<key>%';
\end{lstlisting}

{\bf 检索领域}
\begin{lstlisting}[language=SQL]
select name, direction
from Area
where name like '%<key>%';
\end{lstlisting}


\subsection{排行榜}

在ranklist.html中,我们对所有的学校进行了排名,根据其pub\_cnt,涉及的功能与SQL语句如下:
\begin{itemize}
\item 对所有学校按pub\_cnt降序排名
\begin{lstlisting}[language=SQL]
select name, homepage, pub_cnt
from Institution
order by pub_cnt DESC;
\end{lstlisting}
\item 统计各所学校的教职工人数
\begin{lstlisting}[language=SQL]
select I.name, count(*)
from Institution as I, Scholar as S
where I.name = S.affiliation
group by I.name;
\end{lstlisting}
\item 根据所选领域,筛选出每所学校相应的老师
\begin{lstlisting}[language=SQL]
select I.name, S.name
from Institution as I, Scholar as S
where I.name = S.affiliation and exists (
select *
from Scholar_Paper as SP, Paper as P, Conference_Area as CA
where SP.scholar_name = S.name and SP.paper_title = P.pid and P.conf_id = CA.conf_id and CA.area in :area_list
)
group by I.name;
\end{lstlisting}
\end{itemize}
注:受查询效率所限,我们没有实现按领域排名的功能,实际上这个功能的查询语句可以如下表示:
\begin{lstlisting}[language=SQL]
with Result(ins, cnt) as (
select I.name, count(*)
from Institution as I, Scholar as S, Scholar_Paper as SP, Paper as P, Conferece_Area as CA
where I.name = S.affiliation and S.name = SP.scholar_name and SP.paper_title = P.pid and P.conf_id = CA.conf_id and CA.area in :area_list
group by I.name
)
select I.name, R.cnt
from Institution as I, Result as R
where I.name = R.ins
order by R.cnt DESC;
\end{lstlisting}

\subsection{用户信息}

\subsubsection{用户个人中心}
\noindent
{\bf Profile} 显示用户的基本信息。涉及的SQL语句是:
\begin{lstlisting}[language=SQL]
select username, email, scholar, gender, identity, institution
from Profile
where username=:name;
\end{lstlisting}
{\bf Edit Profile} 提供用户修改个人信息的途径。涉及的SQL语句是:
\begin{lstlisting}[language=SQL]
update Profile
set identity=:identity, gender=:gender, email=:email, institution=:ins,password=:pw
where username=:name;
\end{lstlisting}
{\bf Follow}
\begin{itemize}
\item 显示用户关注的学者、会议、领域及机构。涉及的SQL语句有:
\begin{lstlisting}[language=SQL]
select sch
from User_Scholar
where user=:user;
select ins
from User_Institution
where user=:user;
select area
from User_Area
where user=:user;
select conf
from User_Conference
where user=:user;
\end{lstlisting}
\item 当有关注的学者发表新的论文时,会在此页面上显示提醒。涉及的SQL语句有:
\begin{lstlisting}[language=SQL]
select sch
from User_Scholar
where user=:request.user and new_paper=True;
\end{lstlisting}
\end{itemize}
{\bf My Note}
\begin{itemize}
\item 显示用户写过的所有论文笔记。涉及的SQL语句是:
\begin{lstlisting}[language=SQL]
select title, content, date
from Note
where author=:user;
\end{lstlisting}
\item 当有笔记被其他用户评论时,会在此页面上显示新消息的数量。涉及的SQL语句是:\\
统计总消息数
\begin{lstlisting}[language=SQL]
select count(*)
from Remark, Note
where Remark.note=Note.nid and Note.author=:user and checked=False;
\end{lstlisting}
统计每篇笔记消息数
\begin{lstlisting}[language=SQL]
select count(*)
from Remark
where note=:note and checked=False;
\end{lstlisting}
\end{itemize}
{\bf My Remark} 显示用户写过的所有评论
\begin{lstlisting}[language=SQL]
select content, date
from Remark
where author=:user;
\end{lstlisting}
{\bf Add Paper} 为已经认证过的用户提供添加新论文的途径。用户可以填写论文的标题、年份、会议、超链接及作者名单并提交。当用户已经通过了认证、填写的会议和学者都已经在数据库中存在并且用户是作者之一时,论文才可以添加成功。所涉及的主要SQL语句是:
\begin{itemize}
\item 更新Paper
\begin{lstlisting}[language=SQL]
insert into Paper(title, conf_id, href)
select :title, cid, :href
from Conference
where abbr=:abbr and year=:year;
\end{lstlisting}
\item 更新Schoalr\_Paper
\begin{lstlisting}[language=SQL]
insert into Scholar_Paper(scholar_name, paper_title)
values(:name, :pid);
\end{lstlisting}
\item 更新Scholar\_Area
\begin{lstlisting}[language=SQL]
insert into Scholar_Area(scholar_name, area)
select :name, area
from Conference_Area
where conf_id=:conf;
\end{lstlisting}
\item 更新User\_Scholar,使得关注学者的用户收到提醒
\begin{lstlisting}[language=SQL]
update User_Scholar
set new_paper=True
where sch=:a;
\end{lstlisting}
\item 更新论文作者的pub\_cnt
\begin{lstlisting}[language=SQL]
update Scholar
set pub_cnt = pub_cnt + 1
where name = :name;
\end{lstlisting}
\item 更新论文作者所属学校的pub\_cnt,对于每个论文作者,
\begin{lstlisting}[language=SQL]
update Insititution
set pub_cnt = pub_cnt + (
select count(*)
from Scholar
where Scholar.affiliation = Insitution.name and Scholar.name in :name_list
);
\end{lstlisting}
\end{itemize}

\subsubsection{用户登录注册}

我们实现了用户的登入、登出和注册功能,另外添加了身份认证。登入登出主要依赖Django自带的用户验证系统完成,比较简单,在此略去。

\noindent
{\bf 注册} 用户输入用户名邮箱和密码,经验证没有重名后即可注册成功
\begin{itemize}
\item 检查是否有重名
\begin{lstlisting}[language=SQL]
select conut(*)
from Profile
where username=:username;
\end{lstlisting}
\item 创建新用户
\begin{lstlisting}[language=SQL]
insert into Profile(username, email, password)
values(:username, :email, :password);
\end{lstlisting}
\end{itemize}
{\bf 认证} 注册成功后,系统会筛选姓名包含用户名的学者,让用户选择自己是否是其中之一;经过认证的用户拥有添加论文的额外权限
\begin{itemize}
\item 筛选姓名相似的学者
\begin{lstlisting}[language=SQL]
select name, affiliation
from Scholar
where name like “%:username%”;
\end{lstlisting}
\item 认证后,更新用户信息
\begin{lstlisting}[language=SQL]
update Profile
set scholar=:name, institution=:affiliation
where username=:username;
\end{lstlisting}
\end{itemize}

\subsubsection{论文及笔记}

我们为每篇论文制作了单独的主页,展示论文信息以及这篇论文的所有笔记,登陆后可以添加笔记;点击笔记进入笔记主页,展示了笔记信息和相关评论。同样,登陆后可以添加评论。

\noindent
{\bf 论文主页} 展示论文信息及笔记
\begin{itemize}
\item 筛选笔记
\begin{lstlisting}[language=SQL]
select title, content
from Note
where paper=:pid;
\end{lstlisting}
\item 添加笔记
\begin{lstlisting}[language=SQL]
insert into Note(title, content, author, paper)
values(:title, :content, :user, :paper);
\end{lstlisting}
\end{itemize}
{\bf 笔记主页} 显示笔记、评论,及添加评论
\begin{itemize}
\item 筛选笔记信息及评论
\begin{lstlisting}[language=SQL]
select Note.title, content, date, Paper.title
from Note, Paper
where Note.paper=Paper.pid and Note.nid=:nid;
select title, content, author, date
from Remark
where note=:nid
order by date desc;
\end{lstlisting}
\item 如果笔记作者是当前登录作者,笔记的所有评论设置为已读
\begin{lstlisting}[language=SQL]
update Remark
set checked=True
where note=:nid;
\end{lstlisting}
\item 添加评论
\begin{lstlisting}[language=SQL]
insert into Remark(content, author, note)
values(:content, :user, :note);
\end{lstlisting}
\end{itemize}