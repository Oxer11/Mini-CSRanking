%This file contains the LaTeX code of my laboratory report for my Digital Logic & Computer Design course.
%Author: 张作柏/Zuobai Zhang <17300240035@fudan.edu.cn>

\section{设计细节}

\subsection{排行榜}

\subsection{信息主页}

\subsection{用户信息}

\subsubsection{用户个人中心}
\noindent
{\bf Profile} 显示用户的基本信息。涉及的SQL语句是:
\begin{lstlisting}[language=SQL]
select username, email, scholar, gender, identity, institution
from Profile
where username=:name;
\end{lstlisting}
{\bf Edit Profile} 提供用户修改个人信息的途径。涉及的SQL语句是:
\begin{lstlisting}[language=SQL]
update Profile
set identity=:identity, gender=:gender, email=:email, institution=:ins,password=:pw
where username=:name;
\end{lstlisting}
{\bf Follow}
\begin{itemize}
\item 显示用户关注的学者、会议、领域及机构。涉及的SQL语句有:
\begin{lstlisting}[language=SQL]
select sch
from User_Scholar
where user=:user;
select ins
from User_Institution
where user=:user;
select area
from User_Area
where user=:user;
select conf
from User_Conference
where user=:user;
\end{lstlisting}
\item 当有关注的学者发表新的论文时,会在此页面上显示提醒。涉及的SQL语句有:
\begin{lstlisting}[language=SQL]
select sch
from User_Scholar
where user=:request.user and new_paper=True;
\end{lstlisting}
\end{itemize}
{\bf My Note}
\begin{itemize}
\item 显示用户写过的所有论文笔记。涉及的SQL语句是:
\begin{lstlisting}[language=SQL]
select title, content, date
from Note
where author=:user;
\end{lstlisting}
\item 当有笔记被其他用户评论时,会在此页面上显示新消息的数量。涉及的SQL语句是:\\
统计总消息数
\begin{lstlisting}[language=SQL]
select count(*)
from Remark, Note
where Remark.note=Note.nid and Note.author=:user and checked=False;
\end{lstlisting}
统计每篇笔记消息数
\begin{lstlisting}[language=SQL]
select count(*)
from Remark
where note=:note and checked=False;
\end{lstlisting}
\end{itemize}
{\bf My Remark} 显示用户写过的所有评论
\begin{lstlisting}[language=SQL]
select content, date
from Remark
where author=:user;
\end{lstlisting}
{\bf Add Paper} 为已经认证过的用户提供添加新论文的途径。用户可以填写论文的标题、年份、会议、超链接及作者名单并提交。当用户已经通过了认证、填写的会议和学者都已经在数据库中存在并且用户是作者之一时,论文才可以添加成功。所涉及的主要SQL语句是:
\begin{itemize}
\item 更新Paper
\begin{lstlisting}[language=SQL]
insert into Paper(title, conf_id, href)
select :title, cid, :href
from Conference
where abbr=:abbr and year=:year;
\end{lstlisting}
\item 更新Schoalr\_Paper
\begin{lstlisting}[language=SQL]
insert into Scholar_Paper(scholar_name, paper_title)
values(:name, :pid);
\end{lstlisting}
\item 更新Scholar\_Area
\begin{lstlisting}[language=SQL]
insert into Scholar_Area(scholar_name, area)
select :name, area
from Conference_Area
where conf_id=:conf;
\end{lstlisting}
\item 更新User\_Scholar,使得关注学者的用户收到提醒
\begin{lstlisting}[language=SQL]
update User_Scholar
set new_paper=True
where sch=:a;
\end{lstlisting}
\end{itemize}

\subsubsection{用户登录注册}

我们实现了用户的登入、登出和注册功能,另外添加了身份认证。登入登出主要依赖Django自带的用户验证系统完成,比较简单,在此略去。
\noindent
{\bf 注册} 用户输入用户名邮箱和密码,经验证没有重名后即可注册成功
\begin{itemize}
\item 检查是否有重名
\begin{lstlisting}[language=SQL]
select conut(*)
from Profile
where username=:username;
\end{lstlisting}
\item 创建新用户
\begin{lstlisting}[language=SQL]
insert into Profile(username, email, password)
values(:username, :email, :password);
\end{lstlisting}
\end{itemize}
{\bf 认证} 注册成功后,系统会筛选姓名包含用户名的学者,让用户选择自己是否是其中之一;经过认证的用户拥有添加论文的额外权限
\begin{itemize}
\item 筛选姓名相似的学者
\begin{lstlisting}[language=SQL]
select name, affiliation
from Scholar
where name like “%:username%”;
\end{lstlisting}
\item 认证后,更新用户信息
\begin{lstlisting}[language=SQL]
update Profile
set scholar=:name, institution=:affiliation
where username=:username;
\end{lstlisting}
\end{itemize}

\subsubsection{论文及笔记}

我们为每篇论文制作了单独的主页,展示论文信息以及这篇论文的所有笔记,登陆后可以添加笔记;点击笔记进入笔记主页,展示了笔记信息和相关评论。同样,登陆后可以添加评论。
\noindent
{\bf 论文主页} 展示论文信息及笔记
\begin{itemize}
\item 筛选信息
\begin{lstlisting}[language=SQL]
select scholar_name
from Scholar_Paper
where paper_title=:pid;
select area
from Conference_Area, Paper
where pid=:pid and Paper.conf_id=Conference_Area.conf_id;
\end{lstlisting}
\item 筛选笔记
\begin{lstlisting}[language=SQL]
select title, content
from Note
where paper=:pid;
\end{lstlisting}
\item 添加笔记
\begin{lstlisting}[language=SQL]
insert into Note(title, content, author, paper)
values(:title, :content, :user, :paper);
\end{lstlisting}
\end{itemize}
{\bf 笔记主页} 显示笔记、评论,及添加评论
\begin{itemize}
\item 筛选笔记信息及评论
\begin{lstlisting}[language=SQL]
select Note.title, content, date, Paper.title
from Note, Paper
where Note.paper=Paper.pid and Note.nid=:nid;
select title, content, author, date
from Remark
where note=:nid
order by date desc;
\end{lstlisting}
\item 如果笔记作者是当前登录作者,笔记的所有评论设置为已读
\begin{lstlisting}[language=SQL]
update Remark
set checked=True
where note=:nid;
\end{lstlisting}
\item 添加评论
\begin{lstlisting}[language=SQL]
insert into Remark(content, author, note)
values(:content, :user, :note);
\end{lstlisting}
\end{itemize}